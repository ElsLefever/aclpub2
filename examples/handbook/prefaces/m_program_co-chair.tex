Welcome to the 57th Annual Meeting of the Association for Computational Linguistics! ACL 2019 has been a huge undertaking for the Program Committee (PC), but also very exciting as the conference has set new records on many fronts.
Anticipating a record number of submissions, we recruited the largest PC in the history of ACL. In November 2018, we issued a call for nominations for reviewers, Area Chairs (ACs) and Senior Area Chairs (SACs). By the deadline, we received 851 unique nominations that were used as a starting point for inviting members to the Program Committee. Our committee finally consisted of 2256 members!
In order to manage such a large committee effectively we extended the ACL 2018 practice and created a structure similar to the conferences that have a Senior Program Committee alongside the Program Com- mittee. For the Senior PC, we recruited a relatively large number of Senior Area Chairs (46 SACs, 2–4 to head each area) and Area Chairs (184 ACs, 3–15 per area). We also differentiated between their roles so that SACs assign papers to ACs and reviewers and make recommendations for their area, while ACs each manage a smaller set of papers within the area, lead discussions with reviewers, write meta-reviews and make initial recommendations for a smaller set of papers. This structure also helps to compensate for the problem that our rapidly growing field is suffering from: the lack of experienced reviewers. As ACs focus on a smaller number of papers, they can pay more attention to the review process. As for reviewers, we had many of them this year: 2281. Our 22 thematic areas had 59–319 reviewers each.
We also looked into ways of improving efficiency and the experience for both authors and PC members. In particular, we dropped the paper bidding phase that would require thousands of people to each examine hundreds of abstracts each over a short period of time. Like NAACL 2019, we also dropped the author re- sponse phase that was stressful for authors and time-consuming but not hugely impactful on a larger scale. Finally, we adopted much simpler, streamlined review form, adapted from EMNLP 2018 that encouraged thorough review, but was less laborious for reviewers.
On the submission deadline, we were very glad that we had recruited such large PC and had made all these improvements for increased efficiency: we received 2905 submissions – a 75% increase over ACL 2018 and an all-time record for ACL-related conferences! After the review process, out of the total 2905 submissions (some of which were withdrawn or rejected without review for formatting and policy violations), 660 papers were finally accepted to appear in the conference, resulting in the overall acceptance rate of 22.7%. This is a little lower than the acceptance rate for ACL 2018 (24.9%) or ACL 2017 (23.3%) – yet remarkably similar when we consider the dramatic increase in submissions this year. Among the 660 accepted papers, we have 447 long papers and 213 short papers. As in previous years the acceptance rate is higher for long than for short papers (25.7% vs. 18.3%). Overall, ACL continues to be a very competitive conference. Continuing the tradition, ACL 2019 will also feature presentation of 22 papers that were accepted for publication in the Transactions of the Association for Computational Linguistics (TACL).

ACL 2019 Program Committee Co-Chairs
Anna Korhonen, University of Cambridge
David Traum, University of South California